\documentclass[a4paper,12pt,oneside]{book}

%-------------------------------Start of the Preable------------------------------------------------
\usepackage[english]{babel}
\usepackage{blindtext}
%packagr for hyperlinks
\usepackage{hyperref}
\hypersetup{
    colorlinks=true,
    linkcolor=blue,
    filecolor=magenta,      
    urlcolor=cyan,
}

\urlstyle{same}
%use of package fancy header
\usepackage{fancyhdr}
\setlength\headheight{26pt}
\fancyhf{}
%\rhead{\includegraphics[width=1cm]{logo}}
\lhead{\rightmark}
\rhead{\includegraphics[width=1cm]{logo}}
\fancyfoot[RE, RO]{\thepage}
\fancyfoot[CE, CO]{\href{http://www.e-yantra.org}{www.e-yantra.org}}

\pagestyle{fancy}

%use of package for section title formatting
\usepackage{titlesec}
\titleformat{\chapter}
  {\Large\bfseries} % format
  {}                % label
  {0pt}             % sep
  {\huge}           % before-code
 
%use of package tcolorbox for colorful textbox
\usepackage[most]{tcolorbox}
\tcbset{colback=cyan!5!white,colframe=cyan!75!black,halign title = flush center}

\newtcolorbox{mybox}[1]{colback=cyan!5!white,
colframe=cyan!75!black,fonttitle=\bfseries,
title=\textbf{\Large{#1}}}

%use of package marginnote for notes in margin
\usepackage{marginnote}

%use of packgage watermark for pages
%\usepackage{draftwatermark}
%\SetWatermarkText{\includegraphics{logo}}
\usepackage[scale=2,opacity=0.1,angle=0]{background}
\backgroundsetup{
contents={\includegraphics{logo}}
}

%use of newcommand for keywords color
\usepackage{xcolor}
\newcommand{\keyword}[1]{\textcolor{red}{\textbf{#1}}}

%package for inserting pictures
\usepackage{graphicx}
\usepackage{caption}
\usepackage{subcaption}

%package for highlighting
\usepackage{color,soul}

%new command for table
\newcommand{\head}[1]{\textnormal{\textbf{#1}}}


%----------------------End of the Preamble---------------------------------------


\begin{document}

%---------------------Title Page------------------------------------------------
\begin{titlepage}
\raggedright
{\Large eYSIP2018\\[1cm]}
{\Huge\scshape Flying Sensor Node \\[.1in]}
\vfill
\begin{flushright}
{\large \hspace{0.05cm} Intern : Abheet Verma \\}
{\large Intern \hspace{0.05cm} : Chirag Shah \\}
\hfill \linebreak
{\large Mentor : Simranjeet Singh \\}
{\large Mentor : Saurav Shandilya \\}
\hfill \linebreak
{\large Duration of Internship: $ 21/05/2018-06/07/2018 $ \\}
\end{flushright}

{\itshape 2018, e-Yantra Publication}
\end{titlepage}
%-------------------------------------------------------------------------------

\chapter[Project Tag]{Flying Sensor Node}
\section*{Abstract}

Drone and IoT are two emerging technologies having a plethora of applications. Through this project we aim to bridge these technologies to develop an indoor and outdoor environment monitoring system. A drone acts as a sensor node and flies across a room using way points predetermined by the user, collecting the temperature and humidity data using on-board sensors. The sensor data along with the location and time stamp are sent to central IoT server for data visualization and analytics.\\ \\ Different drones were used for indoor and outdoor environments to tackle the unique challenges faced in terms of mobility and localization in each environment. The concept of charging the drone through a landing deck was also explored in order to realize the potential of Flying Sensor Node in real world applications.
   
\subsection*{Completion status}
\subsubsection{Indoor Flying Sensor Node Tasks} 

\begin{itemize}
\item Pluto X on-board sensor and peripheral interfacing using existing APIs
\item STM32f303CB peripheral interfacing using standard peripheral libraries
\item AR Drone navigation in Gazebo simulation for both Whycon and GPS
\item AR Drone communication in Gazebo simulation for IMU values
\item IoT Server for sensor data visualisation and live tracking was created on the control station
\item Establish communication between Drone and IoT Platform in simulation and reality
\item Decawave UWB tag and anchor hardware setup and configuration
\item UWB tag interfacing on Pluto X
\item Pluto X location co-ordinates transmission and reception using Multi Wii Serial Protocol
\item Complementary Filter design for localization data (Needs Improvement)
\item Payload Testing of Pluto X
\item Waypoint planning and navigation with data logging using UWB and Whycon
\item Landing of Pluto X on designated charging area 
\end{itemize}

\subsubsection{Outdoor Flying Sensor Node Tasks} 

\begin{itemize}
\item Quad-Copter assembly and RC remote testing
\item Pixhawk setup and configuration using Mission Planner and Q Ground Control was attempted
\item Pixhawk control using mavros, dronekit and companion computer was attempted
\item Navio 2 setup for quad-copter was done on RPI 3B
\item Quad copter configuration and calibration was done using Mission Planner
\item IoT Server for sensor data visualisation, live tracking and video streaming was created on the RPI 3B
\item Location logging using GPS was implemented 
\item Temperature and humidity value from DHT-22 is communicated to the flight controller and is logged
\item Mounting structure for Sensor node was designed 
\item Battery consumption was analyzed and appropriate power source was selected
\item Fail safes were removed to enable indoor testing
\item Stabilize mode of operation was tested indoors using RC transmitter
\end{itemize}

\section{Hardware parts}

\subsubsection{Pluto X } 

\begin{itemize}
  \item Purpose : Indoor Drone based on STM32f303CB
  \item Vendor  : Drona Aviation 
 \end{itemize}
 
 \subsubsection{RPI 3B } 

\begin{itemize}
  \item Purpose : RPI 3B is the processing unit for the outdoor quad-copter when using Navio 2 
  \item Product Link : \href{https://www.raspberrypi.org/products/raspberry-pi-3-model-b/}{RPI 3B} 
  \item Datasheet : \href{https://cdn.sparkfun.com/datasheets/Dev/RaspberryPi/2020826.pdf}{RPI 3B Datasheet} 
 \end{itemize}
 
 \subsubsection{Arduino Nano } 

\begin{itemize}
  \item Purpose : Arduino Nano is used as sensor node
  \item Product Link : \href{https://robu.in/product/arduino-nano-v3-0-ch340-chip-mini-usb-cable/?gclid=CjwKCAjw4PHZBRA-EiwAAas4ZnkY1Uyx1regBeSCiAQphWNPrAL1DfNjwFlLTFBN2IuJMkXe9YMaixoCQcsQAvD_BwE}{Arduino Nano} 
  \item Datasheet : \href{https://www.arduino.cc/en/uploads/Main/ArduinoNanoManual23.pdf}{Arduino Nano Datasheet} 
 \end{itemize}

 \subsubsection{Navio 2 } 

\begin{itemize}
  \item Purpose : Navio 2 is an autopilot hat for RPI
  \item Product Link : \href{https://store.emlid.com/product/navio2/}{Navio 2} 
  \item Product Brief : \href{https://docs.emlid.com/navio2/}{Navio 2 Documentation} 
 \end{itemize}
 
 \subsubsection{Decawave DWM1001 } 

\begin{itemize}
  \item Purpose :Indoor Localization
  \item Product Link : \href{https://www.findchips.com/search/DWM1001?gclid=CjwKCAjw4PHZBRA-EiwAAas4ZtTtjHXX9dnmI99yqhUIe2f_LFBmznETxFEPqzojVR2hLfT5Xpz6JRoCmh8QAvD_BwE&gclsrc=aw.ds}{DWM 1001} 
  \item Product Brief : \href{https://www.decawave.com/products/dwm1001-module}{DWM 1001 Resources} 
 \end{itemize}
 
 \subsubsection{LiPo Battery 6200 mAh, 40C \& 11.1 V } 

\begin{itemize}
  \item Purpose : Powering the quad-copter
  \item Product Link : \href{https://robu.in/product/orange-11-1v-6200mah-3s-40c-lipo-battery-pack-xt60-connector/}{Orange Battery} 
  \item Product Brief : \href{https://robu.in/product/orange-11-1v-6200mah-3s-40c-lipo-battery-pack-xt60-connector/}{Orange Battery Specifications} 
 \end{itemize}
 
  \subsubsection{Pixhawk} 
\begin{itemize}
  \item Purpose : Flight controller
  \item Product Link : \href{https://robokits.co.in/drones-quad-hexa-octa-fpv/flight-controllers/pixhawk-px4-2.4.8-32bit-flight-controller-with-imp.-accessories}{Pixhawk} 
  \item Product Brief : \href{https://pixhawk.org/}{Pixhawk Resources} 
 \end{itemize}
 
\subsubsection{Telemetry Radio 433MHz for Navio 2 } 

\begin{itemize}
  \item Purpose : Communication between ground control station and quad-copter 
  \item Product Link : \href{https://store.mrobotics.io/mRo-SiK-Telemetry-Radio-V2-433Mhz-p/mro-433sikv2-mr.htm}{Mrobotics Telemetry Module} 
 \end{itemize}

\subsubsection{Telemetry Radio 433MHz for Pixhawk } 

\begin{itemize}
  \item Purpose : Communication between ground control station and quad-copter 
  \item Product Link : \href{https://robokits.co.in/drones-quad-hexa-octa-fpv/fpv-video-telemetry-osd/433mhz-telemetry-module-pair-for-pixhawk-and-apm-100mw-2km-range}{Pixhawk Compatible Telemetry Module} 
   \end{itemize}
 
 \subsubsection{DHT-22} 

\begin{itemize}
  \item Purpose : Sensor for obtaining temperature and humidity data 
  \item Product Link : \href{https://www.amazon.in/Generic-Digital-Temperature-Humidity-Sensor/dp/B00O8RIYYU}{DHT-22} 
  \item Product Brief : \href{https://www.sparkfun.com/datasheets/Sensors/Temperature/DHT22.pdf}{DHT-22 Datasheet} 
 \end{itemize}
 
  \subsubsection{BRUSHLESS MOTOR SPEED CONTROLLER ESC 30A} 

\begin{itemize}
  \item Purpose : Electronic speed controller for BLDCs
  \item Product Link : \href{https://robokits.co.in/quadrotors-hexarotors-drones/brushless-motors-esc/brushless-motor-speed-controller-esc-30a/}{ESC Specifications} 
 \end{itemize}

\subsubsection{RC BRUSHLESS MOTOR 2212 1000KV WITH SOLDERED BANANA CONNECTOR} 

\begin{itemize}
  \item Purpose : BLDCs with propellers to 
  \item Product Link : \href{https://robokits.co.in/quadrotors-hexarotors-drones/brushless-motors-esc/brushless-motor-speed-controller-esc-30a/}{BLDC Specifications} 
 \end{itemize}

\section{Software used}
\begin{itemize}
 \subsection{Indoor}
  \item Java
  
  For Windows:\href{http://www.oracle.com/technetwork/java/javase/downloads/jdk8-downloads-2133151.html}{Link}\\
  For Ubuntu:\href{http://tipsonubuntu.com/2016/07/31/install-oracle-java-8-9-ubuntu-16-04-linux-mint-18/}{Link}
  
  Use Java 8 only as it is required for Cygnus.

  \item Cygnus IDE
  
  This integrated developement environment is necessary for flashing code to the pluto drone.Install cygnus from \href{https://drive.google.com/drive/folders/12yho1OL4OuOJdStSYlG2r4aEH-reXx16}{here}.Follow the \href{https://github.com/eYSIP-2018/Flying-Sensor-Node/wiki/Setting-Up-Cygnus}{instructions} to setup Cygnus on you system.Enjoy coding!!!!
  
  \item ROS(Indigo)
  
  Robot Operating System (ROS) is robotics middleware (i.e. collection of software frameworks for robot software development).Ros is required for interfacing Pluto and PlutoX.To know more about ROS visit \href{https://en.wikipedia.org/wiki/Robot_Operating_System}{here}.
  
  The main ROS client libraries (C++ and Python) are geared toward a Unix-like system, primarily
because of their dependence on large collections of open-source software. Hence these client libraries
require Linux operating system.\\
You \textbf{must} install the \textbf{ROS-Indigo in Ubuntu 14.04} on your PC/Laptop.

	Follow the installation instructions from \href{http://wiki.ros.org/indigo/Installation/Ubuntu}{here}.
   
   \item pluto\_drone
   
   Metapackage to control the plutodrone via service and topic \href{http://wiki.ros.org/pluto_drone}{wiki} 
   
   
   \item Roslibjs
   
   Roslibjs is the core JavaScript library for interacting with ROS from the browser. It uses WebSockets to connect with rosbridge and provides publishing, subscribing, service calls.\\
   Download link:
   \href{https://static.robotwebtools.org/roslibjs/current/roslib.min.js}{min}
   ::
   \href{https://static.robotwebtools.org/roslibjs/current/roslib.js}{full}
   Source:\href{https://github.com/RobotWebTools/roslibjs}{Github}
   Wiki:\href{http://www.ros.org/wiki/roslibjs/}{Roslibjs}\\
   
  \item Rosbridge\_suite
  
  Rosbridge provides a JSON API to ROS functionality for non-ROS programs. There are a variety of front ends that interface with rosbridge, including a WebSocket server for web browsers to interact with.
  For installation visit this \href{http://wiki.ros.org/rosbridge_suite}{link}.
  \subsection{Outdoor}
  \item Ground Control Station
  
  A ground control station (GCS) is a land- or sea-based control centre that provides the facilities for human control of Unmanned Aerial Vehicles (UAVs or "drones").It may also refer to a system for controlling rockets within or above the atmosphere, but this is discussed elsewhere.
  
  You may install one the below GCS:\\ \\
  Mission Planner::\href{http://ardupilot.org/planner/docs/common-install-mission-planner.html}{Link}(only for Windows)\\
  QGroundControl::\href{https://docs.qgroundcontrol.com/en/getting_started/download_and_install.html}{Link}
  
  \item Python 
  
  Python is an interpreted high-level programming language for general-purpose programming. Created by Guido van Rossum and first released in 1991, Python has a design philosophy that emphasizes code readability, notably using significant whitespace. It provides constructs that enable clear programming on both small and large scales.\\ \\
  We used Python 2.7 for this project.
  \textbf{Always use same python version for every library as it may cause errrors}.\\
  For installation visit \href{https://www.python.org/getit/}{here}.
  
  \item Emlid Image for Raspberry Pi
  
  Raspberry Pi needs an Operating System for operation.A custom image of Raspberry Pi comes with pre installed Ardupilot,Ros and other necessary packages required for automatted vehicles.\\Download the \href{http://files.emlid.com/images/emlid-raspbian-20180525.img.xz}{image} and flash it on a memory card using \href{https://etcher.io/}{Etcher}
  \\
  Run and install Etcher using admin rights.\\Select the archive with Image and drive location.\\Click Flash!!.This may take a few minutes.
  
  \item Motion
  
  Motion is a highly configurable program that monitors video signals from many types of cameras. It was used for displaying live video feed on the web server coming from RPi Camera mounted on the Raspberry Pi. Setting it up  on raspberry pi can be understood from this \href{https://pimylifeup.com/raspberry-pi-webcam-server/}{link}. If the video doesn't start as soon as the server starts and a gray screen is visible visit this \href{https://raspberrypi.stackexchange.com/questions/60669/unable-to-open-video-device
}{discussion}
  
 
  \item Flask is a micro web framework written in Python. It is classified as a microframework because it does not require particular tools or libraries. It has no database abstraction layer, form validation, or any other components where pre-existing third-party libraries provide common functions. However, Flask supports extensions that can add application features as if they were implemented in Flask itself.Download flask from \href{http://flask.pocoo.org/docs/0.12/installation/}{here}.
  
\end{itemize}

\section{Assembly of hardware}
The assembly for both indoor and outdoor drones is explained in this section. 

\subsection*{Indoor Flying Sensor Node} 

\subsubsection*{DWM1001 Module Setup}
\begin{itemize}
  \item One module is used as a tag on the drone and the connections are as shown in the figure
  \item Four modules are used as anchors on the wall and only require a power source of 3.3V 
 \end{itemize}
\includegraphics[width=15cm, height=8cm]{Pinout_DWM.png}
\begin{itemize}
  \item The modules are configured using the official \href{https://play.google.com/store/apps/details?id=com.decawave.argomanager}{Decawave Android App}
  \item The instructions for the same are explained in the above application in great detail. 
 \end{itemize}

\subsubsection*{Pluto X}
\begin{itemize}
  \item After attaching the tag on the gpio shield of Pluto X, place the shield on the Pluto X as shown in the figure, "Pluto X Shield Setup"
  \item The pin mapping of the shield with respect to the STM32f303CB micro-controller on Pluto X is as shown in the table below \\ \\
  \includegraphics[width=15cm, height=6cm]{pinout.PNG}

\begin{figure}
\centering
\begin{subfigure}{.6\textwidth}
  \centering
  \includegraphics[width=.8\linewidth]{PlutoX_without_shield.jpg}
  \caption{Pluto X without shield}
  \label{fig:sub1}
\end{subfigure}%
\begin{subfigure}{.5\textwidth}
  \centering
  \includegraphics[width=.8\linewidth]{Pluto_w_shield.jpg}
  \caption{Pluto X with shield}
  \label{fig:sub2}
\end{subfigure}
\caption{Pluto X Shield Setup}
\label{fig:test}
\end{figure}
\item A whycon marker is also added on top of the Pluto X as shown in the figure below to enable landing on charging area using an overhead camera \\ \\

\centering
\includegraphics[width=9cm, height=9cm]{whycon.jpg}
\end{itemize}

\subsection*{Outdoor Flying Sensor Node}
\subsubsection*{Auto Pilot Assembly}
\begin{itemize}
\item Raspberry Pi, Navio 2, GPS antenna, Telemetry transreceiver, battery monitor, remote receiver through ppm encoder and electronic speed controller inputs are connected as shown \href{https://docs.emlid.com/navio2/ardupilot/hardware-setup/}{here}
\item The other parts of the drone such as motors and propellers are assembled in accordance to this \href{http://ardupilot.org/copter/docs/connect-escs-and-motors.html}{guide} 
\item A three cell 11.1V, 6200mAh, 40C lithium polymer battery is used to power the drone
\item The completely assembled quad-copter is as shown in the image below \\ \\
\centering
\includegraphics[width=9cm, height=9cm]{big_drone.jpg}
\end{itemize}

\begin{itemize}
\item The sensor node comprising of the arduino nano, DHT-22 and usb connector is as shown in the figure below.
\item The sensor node needs to be plugged in any of the usb ports of the raspberry pi 3b on the drone.
\end{itemize}
\begin{figure}
\centering
\begin{subfigure}{.6\textwidth}
  \centering
  \includegraphics[width=.8\linewidth]{sensor_SV.jpg}
  \caption{Sensor Node Side View}
  \label{fig:sub11}
\end{subfigure}%
\begin{subfigure}{.6\textwidth}
  \centering
  \includegraphics[width=.8\linewidth]{sensor_top.jpg}
  \caption{Sensor node top View}
  \label{fig:sub21}
\end{subfigure}
\caption{Sensor Node Module}
\label{fig:test1}
\end{figure}


\section{Software and Code}
\href{https://github.com/eYSIP-2018/Flying-Sensor-Node/}{Github link} for the repository of code

\subsection*{Outdoor}
\begin{itemize}
\item \textbf{Arduino Code} - Contains code to interface DHT-22 using arduino nano on a Raspberry PI.
\item \textbf{Drone-Server} - Contains server for drone.Copy the folder on Raspberry PI.
\item \textbf{Ardupilot Setup} - \href{https://docs.emlid.com/navio2/}{link}

\end{itemize}

\subsection*{Indoor}
\begin{itemize}
\item \textbf{Documents} - Contains API for interfacing PlutoX board.
\item \textbf{Codes} - Contains testing codes for PlutoX and RF.
\item \textbf{Pluto-X Navigation-Whycon,Final,Navigation} - Different Navigation scripts for PlutoX
\item \textbf{PlutoX Firmware} - Firmware Changes for MSP protocol
\item \textbf{Server} - Establishing server for logging data and controlling drone remotely
\item \textbf{pluto\_drone} - Contains changes in navigation packages 
\end{itemize}


\subsection*{Use and Demo}
\subsubsection*{Outdoor}
\begin{itemize}
\item After setting up the raspberry with the custom image and configuring the quad-copter using mission planner do the following :-
\item \textbf{Step 1} - Change directory to Drone-Server
\item \textbf{Step 2} - Run command "sudo modprobe bcm2835-v4l2" to get video on web server.
\item \textbf{Step 3} - Run command "sudo python SensorLog.py" to get real time sensor data i.e. temperature and humidity
\item \textbf{Step 4} - Run command "sudo python GPS.py" to get real time sensor data i.e. GPS (only if the drone is outdoors)
\item \textbf{Step 5} - Run command "sudo python app.py" to start the server
\item \textbf{Step 6} - If the IP address of the raspberry pi is 192.168.43.238 open a browser on a device connected to the network and enter the URL 192.168.43.238:5010 to get access the IoT platform.

\end{itemize}

\subsection*{Indoor}
After setting the DWM1001 module,whycon on the drone and the overhead camera follow the below steps.
\begin{itemize}
\item \textbf{Step 1} - Make changes to the serial\_msp.cpp and flash the firmware.
\item \textbf{Step 2} - Add changed files to pluto\_drone metapackage
\item \textbf{Step 3} - Run the data\_via\_rosservice.py file to get coordinates and other data
\item \textbf{Step 4} - Run the launch file and on another terminal run the navigation script.
\item \textbf{Step 5} - Run the rosbridge\_websocket launch and open comm.html in browser to control the drone and get data
\item \textbf{PS} - Play with the code to understand it better.
\end{itemize}

\href{https://www.youtube.com/watch?v=DnNQvGCwf-I}{Youtube Link} of demonstration video 

\section{Future Work}
\subsection{Outdoor}
\begin{itemize}
\item Manual and Autonomous testing of drone in outdoor environment using Ground Control Station. 
\item Full fledged IOT server for large scale use using internet.
\item Buildng Charging Dock for the drone.
\item Using Dronekit and similar libraries for controlling drone using server.
\end{itemize}
\subsection{Indoor}
\begin{itemize}
\item Building a Charging Dock for PlutoX.
\item External Sensor(Temperature,Humidity) Interfacing for PlutoX.
\item Designing a better noise reduction filter so that navigation is much smoother.
\item On-board Computing of path to reduce latency and improve performance.
\end{itemize}

\section{Bug report and Challenges}
\subsection{Outdoor}
\begin{itemize}
\item No heartbeat received from the FCU.
\item Arming the quadcopter without GPS fix.
\item Bad logging error due to SD card.
\item Telemetery issues with Pixhawk flight controller
\end{itemize}
\subsection{Indoor}
\begin{itemize}
\item Understanding the firmware of PlutoX.
\item Reconfiguring the firmware to accept the coordinate values coming from DWM1001.
\item Understanding Multi Wii Serial Protocol of data transfer from drone to control station and vice versa.
\item Reconfiguring the ROS communication package for PlutoX 'plutodrone' to accept new MSP headers.
\item Designing a filter to reduce noise in the coordinates.
\item Navigating drone using PID and then switching dynamically between whycon marker and Ultra Wide Band for better navigation.
\item Fine tuning of PID for smooth control of drone.
\end{itemize}

\begin{thebibliography}{li}
\bibitem{ros1}
Mastering ROS for Robotics Programming - Lentin Joseph
\bibitem{ros2}
Ros tutorials-
\href{http://wiki.ros.org/ROS/Tutorials}{Link}
\bibitem{ros3}
Programming Robots with ROS: A Practical Introduction to the Robot
Operating - Brian Gerkey, Morgan L. Quigley, and William D. Smart
\bibitem{cheat}
\href{http://www.tedusar.eu/files/summerschool2013/ROScheatsheet.pdf}{ROS cheatsheet}
\bibitem{navio}
Navio Documentation-\href{https://docs.emlid.com/navio2/}{link}
\bibitem{linux}
Linux-\href{https://ryanstutorials.net/linuxtutorial/}{Documentation}
\bibitem{python}
Python-\href{https://www.tutorialspoint.com/python/}{Documentation}
\bibitem{msp}
Multi Wii Serial Protocol-\href{http://www.multiwii.com/wiki/index.php?title=Multiwii_Serial_Protocol}{link}
\bibitem{navio2git}
Navio2 drivers-\href{https://github.com/emlid/Navio2}{Github}
\end{thebibliography}


\end{document}

